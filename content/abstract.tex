\begin{abstract}
    On the way to fully autonomously driving vehicles a multitude of challenges have to be overcome. One common problem is the navigation of the vehicle from a start pose to a goal pose in an environment that does not provide any specific structure (no defined lanes). Typical examples of such environments are usually large open spaces, such as parking lots or construction sights. Here the vehicle needs to be navigated safely around obstacles ideally using the optimal path between the start and the goal state.
    
    The work conducted throughout this master's thesis focuses on the development of a suitable path planning algorithm for the Research Concept Vehicle (RCV) of the Integrated Transport Research Lab (ITRL) at KTH The Royal Institute of Technology, in Stockholm, Sweden.
    
    The development of the path planner requires more than just the pure algorithm, as the code needs to be tested and respective results evaluated. In addition, the resulting algorithm needs to be wrapped in a way that it can easily be deployed and interfaced with different other systems on the research vehicle. Thus the thesis also gives insights into ways of achieving real-time capabilities necessary for experimental testing as well as on how to setup an visualization environment for simulation and debugging.
    
\end{abstract}