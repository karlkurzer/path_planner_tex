\begin{abstract}
On the way to fully autonomously driving vehicles a multitude of challenges have to be overcome. One common problem is the navigation of the vehicle from a start pose to a goal pose in an environment that does not provide any specific structure (no preferred ways of movement). Typical examples of such environments are parking lots or construction sites; in these scenarios the vehicle needs to navigate safely around obstacles ideally using the optimal (with regard to a specified parameter) path between the start and the goal pose.

The work conducted throughout this master's thesis focuses on the development of a suitable path planning algorithm for the Research Concept Vehicle (RCV) of the Integrated Transport Research Lab (ITRL) at KTH Royal Institute of Technology, in Stockholm, Sweden.

The development of the path planner requires more than just the pure algorithm, as the code needs to be tested and respective results evaluated. In addition, the resulting algorithm needs to be wrapped in a way that it can be deployed easily and interfaced with different other systems on the research vehicle. Thus the thesis also tries to gives insights into ways of achieving real-time capabilities necessary for experimental testing as well as on how to setup a visualization environment for simulation and debugging.
\end{abstract}

\begin{acknowledgements}
As this this was written at the ITRL (Integrated Transport Research Lab) I had the opportunity to work together with a variety of people from different fields. I thank my supervisors Mikael Nybacka, John Folkesson and Jonas Mårtensson for their confidence in me and the chance to work on this interesting and seminal topic.

Furthermore I want to thank Andreas Högger for his suggestion of using \texttt{C++} in combination with ROS and his continuous support with both, without ROS the integration into the RCV architecture would have been much more cumbersome. I also want to thank the invaluable asset -- Rui Oliveira for always giving me the opportunity to discuss concepts, ideas and foster understanding.

Additional thanks go to Niclas Evestedt an Erik Ward, who have clarified key questions throughout the thesis and have helped integrate the actual code on the vehicle.

Special thanks go to Moritz Werling (BMW) who not only initially pointed me in the direction of the topic, but also gave valuable insights with regard to the algorithm in general. I also want to thank Marcello Cirillo (Scania), who not only mentioned important implementation details of search algorithms, but asked critical questions that improved the overall quality.
\end{acknowledgements}