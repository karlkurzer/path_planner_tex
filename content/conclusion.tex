\chapter{Conclusion}
While the research in the area of autonomous driving continuous to grow, the DARPA Grand Challenge and especially the Urban Challenge have marked important milestones, greatly propelling the development. Hence it is not astonishing that companies heavily engaged in autonomous driving, such as Alphabet (formerly Google) or Uber recruited a large part of past participants of these challenges.

Since there are many ways to solve navigation problems of this kind the choice might seem arbitrary. While the team from CMU and winner of the DUC used a lattice planner with D* \cite{Ferguson.2008b,Likhachev.2005}, the MIT team used an heavily modified version of RRT both for structured as well as unstructured driving \cite{Kuwata.2008}.

The hybrid A* algorithm is yet another approach. It is a fast planner used for planning in unstructured environments by the Standford team, developed by Dolgov, Thrun, Montemerlo (Google Self-Driving Cars), Diebel (Cedar Lake Ventures), participating with Junior in the DUC.
This thesis has analyzed HA* thoroughly and developed a similar version for the KTH RCV in \texttt{C++} and ROS.

The resulting hybrid A* planner addresses the problem of finding a solution to the problem described in \fref{sec:problemDescription} properly. The planner models the non-holonomic nature of the vehicle in all stages of the process, vertex expansion, heuristic estimates, as well as analytic expansion. Thus, the most important characteristic of the paths is given--they are driveable. 

The HA* planner solves a challenging problem in an elegant manner. HA* is fast, as it reduces the search space with the help of well informed heuristics, allowing it to converge to the goal quickly. Based on the initial solution the local smoothing attains a solution approaching the global optimum. Ideal scenarios for the planner are slow speed driving in unstructured environments. An example of that might be navigating parking lots as well as automated parking and valet parking.

The source code for this project is publicly available and can be found here, \url{https://github.com/karlkurzer/path_planner}; an additional documentation can be found here \url{http://karlkurzer.github.io/path_planner}.