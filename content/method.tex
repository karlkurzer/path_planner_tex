\chapter{Method}
The second part of this thesis shall give an overview over the actual method used, its implementation as well as the results. This chapter will talk about the method used to solve the navigation problem.

The navigation of a mobile robot in the real as opposed to a discretized world, boils down to a complex, continuous-variable optimization problem. While there are a great variety of optimal and fast planners for discrete space, these planners often produces solutions that are not suitable for non-holonomic vehicles as they are not smooth and do not incorporate the vehicle constraints properly.
Other approaches, such as RRT's do produce continuous solutions, but inherently come with disadvantages of being highly nondeterministic or finding solutions that are far from the optimum. The problem regarding the optimality of RRT* has been solved elegantly by Karman et al\cite{Karaman.2011}.

Building on A*, hybrid A* can be seen as the extension of an optimal, deterministic and complete algorithm. Hybrid A* is resolution complete, deterministic and due to the usage of admissible heuristics as well as path post processing the solutions produced are in the neighborhood of the global optimum.

The hybrid A* planner can be split in three distinct parts. The hybrid A* search that incorporates the vehicles constraints, the heuristics that make the search a well informed, allowing for fast convergence; and the path smoothing that improves the found solution using gradient descent.

\chapter{Hybrid A* Planner}

\section{Vehicle Constraints}

\section{Algorithm}

Under the assumption that the steering actions are either maximum steering right, no steering, or maximum steering left the arc length can be simply expressed as $r\abs{x_\theta- x'_\theta}$, $r$ being the minimum turning radius of the vehicle. 

Calculation of the new states $x'$, with the initial state $x$ consisting of :\\
$x_x, x_y, x_\theta$ the turning radius $r$, the turning angle $\theta$ and distance $d$ as well as the unit vector $e = \left(\begin{smallmatrix}0\\1\end{smallmatrix}\right)$
\begin{description}
  \item[left] \hfill \\
  $r_\theta = x_\theta + \frac{\pi}{2}$\\
  $x'_\theta = x_\theta + \theta$
  \item[right] \hfill \\
  $r_\theta = x_\theta - \frac{\pi}{2}$\\
  $x'_\theta = x_\theta - \theta$
  \item[left and right] \hfill\\
  $r_x = r(e_x\cos(r_\theta) - e_y\sin(r_\theta)) + x_x$\\
  $r_y = r(e_x\sin(r_\theta) - e_y\cos(r_\theta)) + x_y$\\
  $x'_x = (x_x-r_x)\cos\theta - (x_y-r_y)\sin\theta + r_x$\\
  $x'_y = (x_x-r_x)\sin\theta - (x_y-r_y)\cos\theta + r_y$
  \item[straight] \hfill \\
  $x'_x = x_x - d\sin(x_\theta)$\\
  $x'_y = x_y + d\cos(x_\theta)$
\end{description}

\section{Heuristics}
underestimating distance, thus nodes expanded at the beginning are getting better in the end, when the real cost is being calculated
\subsection{Constrained Heuristic}
A constrained heuristic takes the characteristics of the vehicle into account.
\subsubsection{Dubin's Paths}
Shoot for the goal
\subsection{Unconstrained Heuristic}
An unconstrained heuristic neglects the characteristics of the vehicle.
\subsubsection{Two Dimensional A*}

\section{Collision Checking}

\section{Path Smoothing}

\section{Path Analysis}

% PATH SMOOTHING
\chapter{Path Smoothing}

\section{Cost Function}
\subsection{Obstacle Term}

This term penalizes collisions with obstacles. For all nodes $\bldx_i$ where $|\bldx_i - \bldo_i| \leq d_{obs}^{\lor}$ the cost $P_{vor}$ is defined. It based on the the distance to the next obstacle.

\begin{equation}
P_{obs} = w_{obs} \displaystyle\sum_{i=1}^{N} \sigma_{obs}(|\bldx_i-\bldo_i|-d_{obs}^{\lor})
\end{equation}

\begin{itemize}
\item $\bldo_i$ the location of the closest obstacle to $\bldx_i$
\item $\bldx_i$ the location of a vertex on the path
\item $d_{obs}^{\lor}$  the maximum distance obstacles affect the cost
\item $w_{obs}$ the obstacle weight
\item $\sigma_{obs}$ quadratic penalty function
\end{itemize}

\subsubsection{Gradient}

\begin{equation}
\frac{\partial \sigma_{obs}}{\partial \bldx_i} = \frac{2(|\bldx_i-\bldo_i|-d_{obs}^{\lor})\bldx_i-\bldo_i}{|\bldx_i-\bldo_i|}
\end{equation}

\subsection{Curvature Term}
This term upper-bounds the instantaneous curvature of the path at every node, ensuring driveability.

\begin{equation}
P_{cur} = w_{cur} \displaystyle\sum_{i=1}^{N-1} \sigma_{cur}\left(\frac{\Delta\phi_i}{|\Delta\bldx_i|} - \kappa_{max}\right)
\end{equation}

\begin{itemize}
\item $\Delta\bldx_i = \bldx_i - \bldx_{i-1}$ the displacement vector at the vertex $\bldx_i$
\item $\Delta\phi_i$ = $\cos^{-1}\frac{\bldx_{i}\cdot\bldx_{i+1}}{|\bldx_{i+1}||\bldx_{i+1}| }$ the change in tangential angle at the vertex $\bldx_i$
\item $\kappa_{max}$ the maximum allowable curvature
\item $w_{cur}$ the curvature weight
\item $\sigma_{cur}$ quadratic penalty function
\end{itemize}

\subsection{Gradients}

\begin{equation}
\frac{\partial \kappa_i}{\partial \bldx_i} = \frac{1}{|\Delta\bldx_i|}\frac{\partial\Delta\phi_i}{\partial\cos\Delta\phi_i}\frac{\partial\cos\Delta\phi_i}{\partial\bldx_i}-\frac{\Delta\phi_i}{{\Delta\bldx_i}^2}\frac{\partial\Delta\bldx_i}{\partial\bldx_i}
\end{equation}

\begin{equation}
\frac{\partial \kappa_i}{\partial \bldx_{i-1}} = \frac{1}{|\Delta\bldx_i|}\frac{\partial\Delta\phi_i}{\partial\cos\Delta\phi_i}\frac{\partial\cos\Delta\phi_i}{\partial\bldx_{i-1}}-\frac{\Delta\phi_i}{{\Delta\bldx_i}^2}\frac{\partial\Delta\bldx_i}{\partial\bldx_{i-1}}
\end{equation}

\begin{equation}
\frac{\partial \kappa_i}{\partial \bldx_{i+1}} = \frac{1}{|\Delta\bldx_i|}\frac{\partial\Delta\phi_i}{\partial\cos\Delta\phi_i}\frac{\partial\cos\Delta\phi_i}{\partial\bldx_{i+1}}
\end{equation}

\subsection{Smoothness Term}
This term is a measure of the smoothness of the path.

\begin{equation}
P_{smo} = w_{smo} \displaystyle\sum_{i=1}^{N-1} (\Delta\bldx_{i+1} - \Delta\bldx_i)^2
\end{equation}

\begin{itemize}
\item $\Delta\bldx_i = \bldx_i - \bldx_{i-1}$ the displacement vector at the vertex $\bldx_i$
\item $w_{smo}$ the smoothness weight
\end{itemize}

\subsection{Voronoi Term}
This term guides the path away from obstacles. For $d_{obs} \leq d_{vor}^{\lor}$ the cost $P_{vor}$ is defined. It is based on the position of the node in the Voronoi field.

\begin{equation}
P_{vor} = w_{vor} \displaystyle\sum_{i=1}^{N} \left(\frac{\alpha}{\alpha + d_{obs}(x,y)}\right)\left(\frac{d_{vor}(x,y)}{d_{obs} + d_{vor}(x,y)}\right)\left(\frac{(d_{obs}(x,y) - d_{vor}^{\lor})^2}{(d_{vor}^{\lor})^2}\right)
\end{equation}

\begin{itemize}
\item $d_{obs} > 0$ the distance to the nearest obstacle
\item $d_{edg} > 0$ the distance to the nearest edge of the GVD
\item $d_{vor}^{\lor}$ the maximum distance obstacles affect the Voronoi potential
\item $\alpha > 0$ controls the falloff rate of the field
\item $w_{vor}$ the Voronoi weight
\end{itemize}

\subsubsection{Gradients}

\begin{equation}
\frac{\partial d_{obs}}{\partial \bldx_i} = \frac{\bldx_i-\bldo_i}{|\bldx_i-\bldo_i|}
\end{equation}

\begin{equation}
\frac{\partial d_{edg}}{\partial \bldx_i} =\frac{\bldx_i-\blde_i}{|\bldx_i-\blde_i|}
\end{equation}

\begin{equation}
\frac{\partial\rho_{vor}}{\partial d_{obs}} = \frac{\alpha d_{edg}\left(d_{obs}-d_{vor}^{\lor}\right)\left(\left(d_{edg}+2d_{vor}^{\lor}+\alpha\right) d_{obs}+\left(d_{vor}^{\lor}+2\alpha\right)d_{edg}+\alpha d_{vor}^{\lor}\right)}{{d_{vor}^{\lor}}^2\left(d_{obs}+\alpha\right)^2\left(d_{obs}+d_{edg}\right)^2}
\end{equation}

\begin{equation}
\frac{\partial\rho_{vor}}{\partial d_{edg}} =  \frac{\alpha d_{obs}\left(d_{obs}-d_{vor}^{\lor}\right)^2}{{d_{vor}^{\lor}}^2\left(d_{obs}+\alpha\right)\left(d_{edg}+d_{obs}\right)^2}
\end{equation}

\section{Gradient Descent}

%\begin{algorithm}
%    \caption{Gradient Descent}\label{alg:gradientDescent}
%    \begin{algorithmic}[1]
%        \Require $x_s \cap x_g \in X$
%        \State $O = \emptyset$
%        \State $C = \emptyset$
%        \State $Pred(x_s) \gets null$
%        \State $O.push(x_s)$
%        \While{$O \neq \emptyset$}
%            \State $x \gets$ O.pop()
%            \State $C.push(x)$
%                \For {$u \in U(x)$}
%                    \State $x_{succ} \gets f(x,u)$
%                    \If {$x_{succ} \notin C$}
%                        \If {$x_{succ} \notin O$}
%                            \State $Pred(x_{succ}) \gets x$
%                            \If{$x_{succ} = x_g$}
%                                \State \textbf{return} $x_{succ}$
%                            \EndIf
%                            \State $O.push(x_{succ})$
%                        \EndIf
%                    \EndIf
%                \EndFor
%        \EndWhile
%        \State \textbf{return} null
%    \end{algorithmic}
%\end{algorithm}

\begin{algorithm}
    \caption{Gradient Descent}\label{alg:gradientDescent}
    \begin{algorithmic}[1]
    \State $maxIterations = 1000$
    \State $iterations = 0$
        \While{$iterations < maxIterations$}
        \State $iterations \gets iterations + 1$
        \EndWhile
        \State \textbf{return} null
    \end{algorithmic}
\end{algorithm}